% Created 2013-08-18 Sun 17:53
\documentclass[bigger]{beamer}
\usepackage[utf8]{inputenc}
\usepackage[T1]{fontenc}
\usepackage{fixltx2e}
\usepackage{graphicx}
\usepackage{longtable}
\usepackage{float}
\usepackage{wrapfig}
\usepackage{soul}
\usepackage{textcomp}
\usepackage{marvosym}
\usepackage{wasysym}
\usepackage{latexsym}
\usepackage{amssymb}
\usepackage{hyperref}
\tolerance=1000
\mode<beamer>{\usetheme{Madrid}}
\usepackage{minted}
\usemintedstyle{emacs}
\usepackage{textpos}
\addtobeamertemplate{frametitle}{}{%
\begin{textblock*}{100mm}(.89\textwidth,-1cm)
\includegraphics[height=0.9cm]{figs/tamuc-logo.eps}
\end{textblock*}}
\title[CSci538: Machine Learning]{Machine Learning and Data Analysis with Python}
\institute[TAMUC]{\includegraphics[height=0.9cm]{figs/tamuc-logo.eps}}
\providecommand{\alert}[1]{\textbf{#1}}

\title{Machine Learning and Data Analysis with Python}
\author{Derek Harter}
\date{2013-08-18}
\hypersetup{
  pdfkeywords={},
  pdfsubject={},
  pdfcreator={Emacs Org-mode version 7.8.11}}

\begin{document}

\maketitle

\begin{frame}
\frametitle{Outline}
\setcounter{tocdepth}{3}
\tableofcontents
\end{frame}
\section{Introduction to Python}
\label{sec-1}
\begin{frame}
\frametitle{Additional Resources}
\label{sec-1-1}

The following are additional resources, all free and available online, that you should use to learn Python.

\begin{itemize}
\item \href{http://www.greenteapress.com/thinkpython}{Think Python: How to think like a computer scientist}  \href{http://www.greenteapress.com/thinkpython}{http://www.greenteapress.com/thinkpython}
     A free but actually professionally done and published textbook.
\item \href{https://developers.google.com/edu/python}{Google Developers Python Class} \href{https://developers.google.com/edu/python}{https://developers.google.com/edu/python}
     A short course from Google, but has a good set of videos to cover the basics.
\item \href{http://software-carpentry.org/v4/python/index.html}{Software Carpentry Python Lectures} \href{http://software-carpentry.org/v4/python/index.html}{http://software-carpentry.org/v4/python/index.html}
     Well done video lectures part of a larger course on scientific software development.
\end{itemize}
\end{frame}
\begin{frame}[fragile]
\frametitle{Declaring Variables}
\label{sec-1-2}
\begin{columns}
\begin{column}{0.6\textwidth}
%% variables
\label{sec-1-2-1}

\begin{itemize}
\item Python is a high-level interpreted language.
\item Python does not force you to declare variable types.
\item Type is inferred from assigned value.
\item Python manages memory for you, will garbage collect unreferenced data.
\end{itemize}
\end{column}
\begin{column}{0.4\textwidth}
\begin{block}{Variable Declaration}
\label{sec-1-2-2}


\begin{minted}[]{python}
x = 1
y = x + 3
print x, y
print type(x)
\end{minted}

\begin{verbatim}
 1 4
 <type 'int'>
\end{verbatim}
\end{block}
\end{column}
\end{columns}
\end{frame}
\begin{frame}[fragile]
\frametitle{Operations on Variables}
\label{sec-1-3}
\begin{columns}
\begin{column}{0.6\textwidth}
%% operators
\label{sec-1-3-1}

\begin{itemize}
\item Python includes all of the arithmetic and boolean operations with same syntax as C, Java, etc.
\item Arithmetic operators use standard order of precedence: () ** * / \% + -
\item Boolean operators: == != < > <= >=
\end{itemize}
\end{column}
\begin{column}{0.4\textwidth}
\begin{block}{Operators Example}
\label{sec-1-3-2}


\begin{minted}[]{python}
x = (3 + 5) * 2 ** 3
print x
print x <= 5
\end{minted}

\begin{verbatim}
 64
 False
\end{verbatim}
\end{block}
\end{column}
\end{columns}
\end{frame}
\begin{frame}[fragile]
\frametitle{Functions}
\label{sec-1-4}
\begin{columns}
\begin{column}{0.4\textwidth}
%% functions
\label{sec-1-4-1}

\begin{itemize}
\item A function is a named sequence of statements that performs a computation.
\item Python uses def to define a new function.
\item All Python functions return results, if you don't specify result using
     return, then None is returned as function value.
\end{itemize}
\end{column}
\begin{column}{0.6\textwidth}
\begin{block}{Function Example}
\label{sec-1-4-2}

\fontsize{6}{7.2}\selectfont

\begin{minted}[]{python}
def sum_ceiling(x, y, z, ceiling):
    """Return the sum of x+y+z if it is less than
    maximum ceiling.  Otherwise return the ceiling"""
    s = x + y + z
    if s < ceiling:
        return s
    else:
        return ceiling

print sum_ceiling(3, 8, 11, 20)
print sum_ceiling(1, 2, 3, 99)
\end{minted}

\begin{verbatim}
 20
 6
\end{verbatim}
\end{block}
\end{column}
\end{columns}
\end{frame}
\begin{frame}[fragile]
\frametitle{Built In Data Structures: Lists}
\label{sec-1-5}

   
\begin{columns}
\begin{column}{0.4\textwidth}
%% lists
\label{sec-1-5-1}

\begin{itemize}
\item Lists are sequences of values.
\item The list values do not have to be of the same type (unlike a C or Java array).
\item Lists are indexed by an integer value, starting at 0.
\item Lists can be changed, values added or removed, etc.
\end{itemize}
\end{column}
\begin{column}{0.6\textwidth}
\begin{block}{List Example}
\label{sec-1-5-2}

\fontsize{6}{7.2}\selectfont

\begin{minted}[]{python}
states = ['Alaska', 'Alabama', 'Texas', 'Mississippi']
print states[0]  # first item in list
print states[1:3] # items 1 up to but not including 3 of list
print states[-1] # last item in list
states[2] = 'California'
print states
\end{minted}

\begin{verbatim}
 Alaska
 ['Alabama', 'Texas']
 Mississippi
 ['Alaska', 'Alabama', 'California', 'Mississippi']
\end{verbatim}
\end{block}
\end{column}
\end{columns}
\end{frame}
\begin{frame}[fragile]
\frametitle{Built In Data Structures: Dictionaries}
\label{sec-1-6}
\begin{columns}
\begin{column}{0.4\textwidth}
%% dictionaries
\label{sec-1-6-1}

\begin{itemize}
\item Dictionaries map an arbitrary key to a value (key->value pair).
\item Dictionaries are mutable, values can be changed, added or removed.
\end{itemize}
\end{column}
\begin{column}{0.6\textwidth}
\begin{block}{Dictionary Example}
\label{sec-1-6-2}

\fontsize{6}{7.2}\selectfont

\begin{minted}[]{python}
phone_number = {'John': '818-922-2381',
                'Susan': '414-938-1923',
                'Ray': 9034541238}
print phone_number['Ray']
phone_number['Alice'] = 8184531923
print phone_number
\end{minted}

\begin{verbatim}
 9034541238
 {'John': '818-922-2381', 'Ray': 9034541238, 'Alice': 8184531923, 'Susan': '414-938-1923'}
\end{verbatim}
\end{block}
\end{column}
\end{columns}
\end{frame}
\begin{frame}[fragile]
\frametitle{Built In Data Structures: Tuples}
\label{sec-1-7}
\begin{columns}
\begin{column}{0.4\textwidth}
%% tuples
\label{sec-1-7-1}

\begin{itemize}
\item Tuples are immutable lists, they can't be changed.
\item We mention because you will run across them early, for example to return 
     multiple values from a function, Python programmers often return a tuple of values.
\end{itemize}
\end{column}
\begin{column}{0.6\textwidth}
\begin{block}{Tuples Example}
\label{sec-1-7-2}

\fontsize{6}{7.2}\selectfont

\begin{minted}[]{python}
def find_min_max(l):
    """Return the minumum and maximum values in the list l"""
    minimum = min(l)
    maximum = max(l)
    return (minimum,maximum)

l,h = find_min_max([9, 8, 2, 11, 42, 10])
print "Minimum was: ", l
print "Maximum was: ", h
\end{minted}

\begin{verbatim}
 Minimum was:  2
 Maximum was:  42
\end{verbatim}
\end{block}
\end{column}
\end{columns}
\end{frame}

\end{document}
